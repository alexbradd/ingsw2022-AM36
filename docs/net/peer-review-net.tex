\documentclass{article}
\usepackage[utf8]{inputenc}

\title{Peer Review 2: protocollo di rete} 
\author{Leonardo Bianconi, Alexandru Gabriel Bradatan, Mattia Busso \\(Gruppo AM36)}

\begin{document}

    \maketitle

    \section{Lati positivi}
        \begin{itemize}
            \item Abbiamo valutato positivamente l'utilizzo di JSON come linguaggio
                di serializzazione: esso permette l'astrazione dal linguaggio di 
                programmazione utilizzato (a differenza dell'uso di Java serializable)
                e permette una lettura e un debug più semplici rispetto a, per esempio,
                un protocollo in \textit{plain-text}.
            \item Il vostro documento risulta molto dettagliato e non ci sono casistiche
                (a nostro avviso) che non sono state considerate. 
        \end{itemize}

    \section{Lati negativi}
        \subsection{Informazioni contenute nella stringa \texttt{"message"}}
            Molti dei messaggi inviati dal server contengono informazioni
            all'interno del valore di "message" (a partire dal
            generico messaggio d'errore \texttt{ERR\_MESSAGE}).
            
            Ricevuto un messaggio dal server del tipo:
            \texttt{\{"message"\="Error!!"+Err\}} la sua gestione
            risulta facile per quanto riguarda l'implementazione del client via
            CLI: il messaggio d'errore può essere semplicemente stampato a schermo,
            ma, per quanto riguarda la GUI, risulterebbe complicato eseguire il
            parsing del messaggio d'errore per modificare elementi grafici in base
            all'errore ricevuto (eventualità che immaginiamo possa verificarsi
            in una partita).    % TODO

            Altri esempi sono i messaggi \texttt{TURN\_NUMBER}, \texttt{PLANNING\_PHASE} e
            \texttt{ACTION\_PHASE}: non disponendo di un attributo caratteristico, lato
            client il parsing è necessario per memorizzare informazioni sullo stato
            della partita (cosa che è necesssaria per lo meno lato GUI).

            Per quanto riguarda i messaggi spediti dal client al server, una chiave
            unica per tipo di messaggio è presente (ci riferiamo, ad esempio a
            \texttt{"num\_of\_players"} in \texttt{TWO\_PLAYERS\_GAME}), anche se
            sarebbe più semplice includere questa informazione nel valore di una chiave
            \texttt{"type"}. In questo modo sarebbe possibile eseguire uno \textit{switch}
            su questo valore, piuttosto che controllare che esista una determinata chiave.

        \subsection{Necessità di richiedere manualmente al server gli aggiornamenti}
            Riteniamo problematico il fatto di non aver incluso un messaggio di
            \textit{update} da parte del server, lasciando questo compito al client.
            In particolare, questa scelta si pone in contrasto con l'architettura presentata
            (che assomiglia in molti aspetti a un'architettura \textit{server-driven}).
            Questa scelta inoltre, complica la gestione degli aggiornamenti lato client:
            in questo caso immaginiamo che i client eseguano \textit{polling} continuo per
            ricevere gli aggiornamenti della board, tramite il messaggio di 
            \texttt{REQ\_SHOW\_BOARD}, soprattutto per quanto riguarda i client in fase
            d'attesa.

        \subsection{Messaggi di update di sotto-parti dell'area di gioco}
            Riteniamo che i messaggi \texttt{REQ\_SHOW\_MY\_ENTRANCE}, \texttt{REQ\_SHOW\_MY\_TABLE},\\
            \texttt{REQ\_SHOW\_ISLANDS}, \texttt{REQ\_SHOW\_CLOUDS} (e relative risposte del server)
            siano superflui, e che basti implementare correttamente il messaggio di update generale
            \texttt{SHOW\_BOARD}. Questo perché assumiamo la rete essere molto veloce e non
            ci preoccupiamo del traffico di messaggi scambiati, né della dimensione degli stessi,
            dunque è possibile implementare il solo messaggio di update globale e inviare sempre 
            quello tra client e server.

    \section{Confronto tra i due protocolli}
        \begin{itemize}
            \item Entrambi i protocolli usano JSON, che riteniamo essere una buona scelta perché
                è un formato di messaggi \textit{language-agnostic} e che permette una maggiore
                comprensione dei messaggi e debug rispetto, per esempio, a una serializzazione
                implementata tramite l'interfaccia Java \textit{Serializable}.
            \item Il nostro protocollo differisce dal vostro per quanto riguarda la lunghezza
                e la frequenza dei messaggi: nella nostra architettura i messaggi inviati dal
                client sono uno per interazione, e gli \textit{update} da parte del server 
                vengono inviati in broadcast e fungono anche da messaggi di \textit{acknowledgement}
                di avvenuto cambiamento dello stato. Questa scelta comporta un ridotto numero di
                messaggi implementati ma una maggiore lunghezza degli stessi.
                Nella vostra architettura, i messaggi sono più frequenti, e spesso singole
                interazioni client-server richiedono una serie più numerosa di messaggi scambiati,
                che riflettono i passaggi eseguiti manualmente dal giocatore all'interno del gioco
                (ad esempio, lo scambio di messaggi relativo all'uso delle carte personaggio).
            \item Abbiamo deciso, a differenza della vostra architettura, di inviare nel messaggio
                di update soltanto i cambiamenti del modello dovuti alla precedente azione da parte
                di un client. Il vostro server, invece, invia tutto lo stato corrente di gioco ad 
                ogni messaggio di \texttt{SHOW\_BOARD} (e simili). In generale, riteniamo che entrambe le soluzioni
                siano valide e che la differenza rifletta principalmente la volontà da parte nostra
                di sviluppare un client più \textit{"thick"}, mentre la vostra architettura risulta
                più \textit{server-heavy}.
        \end{itemize}
        
\end{document}