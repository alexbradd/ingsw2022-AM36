\documentclass[12pt]{article}
\usepackage[utf8]{inputenc}
\usepackage[T1]{fontenc}
\usepackage[italian]{babel}
\usepackage{enumerate}

\title{Peer-Review 1: UML}
\author{Leonardo Bianconi, Alexandru Gabriel Bradatan, Mattia Busso\\Gruppo 36}

\begin{document}

    \maketitle

    In questo documento verrà svolta una valutazione del diagramma UML delle classi
    del gruppo 09. L'analisi sarà suddivisa in 3 parti: individuazione dei lati
    positivi, di quelli negativi e un confronto con l'architettura del progetto
    del nostro gruppo.


    \section{Lati positivi}

    \subsection{Uso di \texttt{Enum} per gli assistenti}

    Gli assistenti sono un'entità che contiene solo dati senza funzionalità.
    Perciò l'utilizzo di \texttt{Enum} è un approccio migliore rispetto a creare
    una classe.

    \subsection{Astrazione per il movimento degli studenti}

    L'utilizzo di una classe astratta \texttt{Place} che rappresenta un'entità
    capace di ricevere o mandare studenti ad altre entità dello stesso tipo è un
    approccio che permette di caratterizzare in modo efficace entità simili tra
    loro come ad esempio \texttt{Cloud}, \texttt{Entrance} e \texttt{Table}.
    Tuttavia, l'implementazione ha anche un difetto, vedasi~\ref{ereditarieta}.

    \subsection{La classe \texttt{Board}}

    L'utilizzo di una classe apposita che contiene tutti le entità che
    modellano oggetti fisicamente presenti sul tavolo di gioco permette di
    ridurre la dimensione della classe \texttt{Game} e di regolare l'accesso
    tramite i metodi esposti.

    \subsection{Uso del pattern \textit{Strategy} per le carte personaggio}

    L'uso del pattern \textit{Strategy} permette di astrarre il comportamento
    delle singole carte personaggio fornendo un'interfaccia unificata.

    \subsection{Uso del pattern \textit{Factory} per la creazione di \texttt{Game}}

    L'uso de pattern \textit{Factory} permette di nascondere il processo di
    creazione di \texttt{Game} all'esterno.


    \section{Lati negativi}

    \subsection{Uso scorretto dell'ereditarietà}\label{ereditarieta}

    \texttt{TwoPlayersGame} e \texttt{ThreePlayersGame} non aggiungono
    comportamento alla superclasse e sarebbero potute essere implementate
    parametrizzando la classe \texttt{Game}.

    La classe \texttt{Place} definisce anche comportamento per la gestione dei
    professori, funzione che non appartiene a tutte le sue sottoclassi (per
    esempio \texttt{Cloud}). Sarebbe stato oppurtuno definire due interfacce:
    una che caratterizza le entità che trattano studenti e un'altra che
    caratterizza quelle che trattano professori.

    \subsection{Uso errato o mancato del pattern \textit{Decorator}}
    La decorazione della classe \texttt{Game} per aggiungere la modalità esperti
    risulta eccessiva: come anche per \texttt{TwoPlayersGame} e
    \texttt{ThreePlayersGame} sarebbe più semplice parametrizzare direttamente
    la classe \texttt{Game}. Il metodo \texttt{drawCharacter()} sarebbe più
    appropriato nella classe \texttt{Board} e chiamato durante
    \texttt{StartGame()}.

    Un caso invece in cui l'utilizzo del pattern sarebbe adeguato è per
    modificare a runtime il comportamento dell'influenza.

    \subsection{\texttt{Coin} è superflua}

    La classe \texttt{Coin} è vuota e potrebbe essere rimpiazzata da un intero.

    \subsection{Carte personaggio definiscono il proprio effetto a runtime}

    Le carte personaggio hanno un comportamento fissato per tutta la durata del
    gioco. Il metodo \texttt{ChooseStrategy()} permetterebbe di cambiare a
    runtime gli effetti delle carte, comportamento che è contro le regole del
    gioco. Unificare \texttt{Character} e \texttt{StrategyFX}
    implementando una classe per ogni carta permetterebbe di ovviare a questo
    problema.

    \subsection{L'astrazione introdotta da \texttt{GamePhase} è insufficiente}

    La classe \texttt{GamePhase} è un'opportunità mancata per implementare un
    pattern \textit{State} che avrebbe potuto esprimere in modo più efficace
    e chiaro la macchina a stati del gioco.


    \section{Confronto tra le architetture}

    Le due architetture differiscono su un punto fondamentale: il gruppo 09
    implementa il pattern MVC utilizzando un controller ``\texttt{thick}'' che
    esegue, oltre alla validazione sintattica dei messaggi del client, anche la
    validazione semantica. La nostra architettura, invece, esegue il controllo
    semantico all'interno del model, assottigliando il controller.

    Due punti di forza dell'architettura del gruppo 09 rispetto alla nostra sono
    i seguenti:

    \begin{enumerate}
        \item L'uso della classe \texttt{Board}
        \item L'uso di \texttt{Enum} per gli assistenti
    \end{enumerate}

    Nella nostra architettura le responsabilità della classe \texttt{Board} sono
    delegate alla classe \texttt{Game}, che però si occupa già della
    progressione della macchina a stati. Separare le due permetterebbe una
    migliore ``separation of concerns'' della classe \texttt{Game}. Inoltre la
    nostra classe \texttt{Assistant} teoricamente permetterebbe la creazione di
    carte con combinazioni di valori non valide. Un \texttt{Enum} permetterebbe
    di definire un insieme di valori legittimi.

\end{document}
